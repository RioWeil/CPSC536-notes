\section{Finishing Up Shor, The Hidden Subgroup Problem}

\subsection*{Final Remarks on Shor's Algorithm}
We looked at the order finding algorithm last time. The runtime was $\tilde{O}((\log N)^2)$ for the classical part and $\tilde{O}((\log N)^3)$ for the quantum part (naively). This came from the repeated squaring of $x(s) = a^s\mod N$. But we can actually do this in $(\log N)^2$ time if we do a Fourier transform instead of doing the ``schoolboy'' multiplication. Then, the complexity of each squaring is just $\tilde{O}(\log N)$ and so all of the squarings takes $\tilde{O}((\log N)^2)$ time.

But last year, Oded Regev actually found a way to do this in $\tilde{O}((\log N)^{1.5})$ time. He reduced the complexity of the quantum part!

Idea: What if instead of choosing $a$, we chose $a, b, c, d$. Consider $x(s_1, s_2, s_3, \ldots s_k) = a_1^{s_1}a_2^{s_2}\ldots a_k^{s_k} \mod N$ where $k$ is to be optimized. This has a period in a higher dimensional space of $\ZZ_k$. For each of the $s_i$, we don't have to go as far to find the period; in a higher-dimensional space the period is shorter, so we can raise the numbers to a smaller power and so this gives us some savings.

\subsection*{Basic Group Theory}
\begin{defbox}{: Groups}
    A group $G = (S, \alpha)$ is defined by a set $S$ and a function $\alpha: S \times S \to S$ such that:
    \begin{enumerate}
        \item (Identity) $\exists e \in S$ such that $\forall g \in S$, $\alpha(g, e) = g = \alpha(e, g)$
        \item (Inverses) $\forall g \in S$, $\exists h \in S$ such that $gh = e = hg$. 
        \item (Associativity) $\forall g, h, k \in S$ such that $\alpha(\alpha(g,h), k) = \alpha(g, \alpha(h, k))$. 
    \end{enumerate}
    We didn't include it in the axiom, but the inverse is unique. The \emph{order} of the group is $\abs{S}$. Sometimes the notation is abused and we write $\abs{G}$. A group $G$ is Abelian if the $\alpha$ operation commutes i.e. $\forall g, h \in S$ we have $\alpha(g, h) = \alpha(h, g)$. 
\end{defbox}

Some examples:
\begin{enumerate}
    \item $G = \mathbb{F}_2^n$, with $\alpha = $ componentwise addition.
    \item $G = \ZZ$, with $\alpha = $ addition.
    \item $G = GL_n(\CC)$ the set of invertible $n \times n$ matrices with entries in $\CC$ (with $\alpha = $ matrix multiplication.)
    \item $G = D_{2n}$, the set of symmetries of a regular $n$-gon. A regular 3-gon is a triangle, a regular 4-gon is a square, a regular 5-gon is a pentagon, and so on. Let's take the square/4-gon as an example. The symmetries are the rotations and reflections $\set{0^\circ = e, 90^\circ = \sigma, 180^\circ = \sigma^2, 270^\circ = \sigma^3, \tau, \tau\sigma, \tau\sigma^2, \tau\sigma^3}$. The group operation is composition of the symmetries. $D_{2n}$ can be thus written as $\set{\sigma^s\tau^a \vert s \in \ZZ_n, a \in \ZZ_2}$. It will later be convenient to use the slightly different representation of the element. We can write the elements as $(s, a)$ where $(s, a)\cdot(t, b) = (s + (-1)^a t, a + b)$. 
\end{enumerate}
1, 2 are examples of Abelian groups, while 3,4 are not. 

\begin{defbox}{: Subgroups}
    Let $G = (S, \alpha)$ be a group. We say $T \subseteq S$ form s a subgroup of $G$ if:
    \begin{enumerate}
        \item $T$ contiains the $e$ of $G$
        \item $T$ is closed under $\alpha$, that is $\forall g, h \in T$ we have $\alpha(g, h) \in T$. 
        \item $T$ is inverse closed, i.e. $\forall g \in T$ we have $g^{-1} \in T$. 
    \end{enumerate}
    This definition defines a group $(T, \alpha\vert_T)$ (where $\alpha\vert_T$ denotes the restriction of the $\alpha$ operation onto $T \times T$ (rather than $S \times S$)).
\end{defbox}

Examples:
\begin{enumerate}
    \item Any group $G$ has the trivial subgroups $\set{e}, G$. 
    \item Suppose $G = \mathbb{F}_2^n$. Then, $\set{a, (0, \ldots, 0)}$ where $0^n \neq a \in \mathbb{F}_2^{n}$. This follows from the fact that $a 0^n = a$, $0^n 0^n = 0^n$, $aa = 0^n$ and so the group is closed, has the identity element, and contains all inverses, as required.
    \item Take $G = \ZZ$. Then, $r\ZZ = \set{r \cdot z \vert z \in \ZZ}$ for $0 \neq r \in \ZZ$ is a subgroup.
    \item For $y \in \ZZ_n$, $T_y \coloneqq \set{(0, 0), (y, 1)}$ is a subgroup of $D_{2n}$, as $(y, 1) \cdot (y, 1) = (0, 0)$. 
\end{enumerate}

\begin{defbox}{: Hidden Subgroup Problem}
    Let $G$ be a group and $\H$ denote a set of subgroups of $G$. Let $\Sigma$ be an alphabet with $\abs{\Sigma} \geq \abs{G}$. The HSP$(G, \H, \Sigma)$ problem is to compute the function $f: D \subseteq \Sigma^G \to \H$ where $x \in D$ if and only if there exists $H \in \H$ with $x(g) = x(h) \iff gH = hH$ where $gH = \coloneqq \set{gh \vert h \in H}$
\end{defbox}

\begin{propbox}{: Quantum query complexity of HSP}
    \label{QComplexity-HSP}
    Let $G$ be a finite group. Let $\H$ be the set of all subgroups of $G$ and let be $\Sigma$ be an alphabet such that $\abs{\Sigma} \geq \abs{G}$. Then, $Q(\text{HSP}(G, \H, \Sigma)) = O((\log \abs{G})^2)$.
\end{propbox}

Example: Simon's problem, with $G = \mathbb{F}_2^n \to O(n^2)$. 

\subsection*{Interlude: Mixed Quantum States}
So far:
\begin{enumerate}
    \item A quantum state is a vector $\ket{\psi} \in \CC^d$
    \item A $\Gamma$-outcome measurement on $\CC^d$ applied to $\ket{\psi}$ results in $i \in \Gamma$ according to some probability $p_i$. The state becomes $\ket{\psi_i}$ if $i =$ measurement outcome.
\end{enumerate}
In Simon's/Period finding problems:
\begin{enumerate}
    \item Either we did not care about the measurement outcome (first measurement)
    \item Or we did care about the distribution on the measurement outcome, but not the distribution on the state it ends up in $\ket{\psi_i}$ (second measurement)
\end{enumerate}

In either case, we didn't care about the distribution on the post-measurement state. We have post measurement state $\ket{\psi_i}, p_i$. There are two ways of doing analysis.
\begin{enumerate}
    \item Naive but hard way: Do a for loop; for each $i \in \Gamma$, analyze the post-processing effect on the $\ket{\psi_i}$, then ``average over the effects''.
    \item Sophisticated (but easier) way: define a single objkect that captures the distribution on the post-measurement states, and analyze the effects on the single object.
\end{enumerate}

What is the mystery single object? Is is the density matrix $\rho \in \CC^{d \times d}$, defined as:
\begin{equation}
    \rho = \sum_{i \in \Gamma}p_i \dyad{\psi_i}{\psi_i}
\end{equation}