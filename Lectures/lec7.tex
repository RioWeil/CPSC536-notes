\section{Analysis of the hypercube and Simon's problem}

\subsection*{The hypercube problem continued}
The hypercube problem continued.  To review, if  we consider the $n!$ paths ($n$ choices of which bit to flip in the first step, then $n-1$, and so on) from $0^n$ to $1^n$, and then consider $OR(\tilde{p}_1 \ldots \tilde{p}_{n!})$ where $\tilde{p}_i = 1$ if the path is present and 0 otherwise. If we do a Grover search over all directed paths, then we have $O(\sqrt{n!}n) \sim O(n^{n/2})$ complexity (via the Sterling approximation) which is worse than the trivial upper bound. Better method:

\begin{enumerate}
    \item We now consider if we go from $0^n$ to $1^n$ (and arrange a pictoral diagram going from left to right in order of increasing Hamming weight), we can classically all edges in the shaded areas i.e. from $0$ up to a certain Hamming weight $\alpha n$, and then from Hamming weight $n - \alpha n$ line to $1^n$. The lines are the sets of points $\set{z \in \set{0, 1}^n \vert \abs{z} = \alpha n}$.  Then we can classically query all edges in the shaded areas, which takes $2\sum_{k=0}^{\alpha n}\binom{n}{k} \cdot \frac{n}{2}$ queries.
    \item Now if we try to decide whether $z^{(1)}$ is connected to $z^{(2)}$ via a directed path. We then have a sub-hypercube $\set{z \in \set{0, 1}^n \vert z^{(1)} \leq z \leq z^{(2)}}$. The dimension of this hypercube is $n/2 - \alpha n$. So the number of edges inside of it is $(n/2 - \alpha/n)2^{n/2-\alpha n}/2$. So by investing $O(2^{n/2-\alpha n})$ (hiding the polynomial in the $*$) his number of (classical) queries we can decide if $z^{(1)}$ is connected to $z^{(2)}$
    \item Now we ask how $z^{(2)}$ could possibly connected to $0^n$. We want to OR over the points on the $\abs{z} = \alpha n$ line, so we Grover search over $\sqrt{\binom{n}{\alpha n}}O^*(2^{n/2-\alpha n})$ possibilities.
    \item So then in total we have $\sqrt{\binom{n}{n/2}}\sqrt{\binom{n}{\alpha n}}O^*(2^{n/2-\alpha n})$ so the overall cost is:
    \begin{equation}
        O^*(\sum_{k=0}^{\alpha n}\binom{n}{k} + \sqrt{\binom{n}{n/2}}\sqrt{\binom{n}{\alpha n}}2^{n/2-\alpha n})
    \end{equation}
    then we choose $\alpha$ optimally. For this we use a Lemma:

    \begin{lembox}{: Binomial Coefficient Bounds}
        Let $n \in \NN$. Then:
        \begin{equation}
            \begin{cases}
                \forall k \in \NN, 1 \leq k \leq \frac{n}{2}, \binom{n}{\leq k} \coloneqq \sum_{i=0}^k \binom{n}{i} \leq 2^{h(k/n)n}
                \\ \forall l \in \NN, 1 \leq l \leq n, \binom{n}{l} \leq 2^{h(l/n)n}
            \end{cases}
        \end{equation}
        where $h:[0, 1] \to [0, 1]$ is the binary entropy defined by:
        \begin{equation}
            h(p) \coloneqq -p\log_2(p) - (1-p)\log_2(1-p)
        \end{equation}
    \end{lembox}
    upper bounding using the Lemma:
    \begin{align}
        \sum_{k=0}^{\alpha n}\binom{n}{k} + \sqrt{\binom{n}{n/2}}\sqrt{\binom{n}{\alpha n}}2^{n/2-\alpha n} &\leq 2^{nh(\alpha n/n)} + \sqrt{2^{nh(1/2)}}\sqrt{2^{nh(2\alpha)}} 2^{n/2-\alpha n}
        \\ &= 2^{n\ln(\alpha)} + 2^{n(1 + \frac{1}{2}h(2\alpha) - \alpha)}
    \end{align}
    so we optimize by setting the exponents to being the same. We can do this using our favourite software device to find $\alpha = 0.397$. This implies the quantum query complexity is $2^{nh(0.397)} = 2^{0.969n}$. Mo

    Can we push this to extremes with more layers/recusive averaging? Currently we have $O^*(2^{0.8615n}$. The best known quantum lower bound is $\Omega^*(2^{0.5n})$ and there has been a 5 year open question to close the gap. Another question si whether we can remove the assumption of QRAM. 

\end{enumerate}


\subsection*{Simon's problem}
Consider $\text{Simon}_n : D \coloneqq D_0 \cup D_1 \subseteq \set{0, 1, \ldots, n-1}^n \to \set{0, 1}$. The setting is similar to the collision problem. Here the disjoint sets $D_0, D_1$ are defined as:
\begin{equation}
    D_0 \coloneqq \set{x \in \set{0, 1, \ldots, n-1}^n \vert \forall s \neq t, x(s) \neq x(t)}
\end{equation}
\begin{equation}
     D_1 \coloneqq \set{x: \set{0, 1}^k \to \set{0,1, \ldots, n-1} \vert \exists a \in \set{0, 1}^k - \set{0^k} \forall s, t \in \set{0, 1}^k, x(s) = x(t) \iff s = t \oplus a}
\end{equation}
where $\text{Simon}_n(D_0) = \set{0}$ and $\text{Simon}_n(D_1) = \set{1}$. 

\begin{propbox}{: Quantum/randomized complexities of Simon's problem}
    $Q(\text{Simon}_n) = O(\log n)$ and $R(\text{Simon}_n) = \Omega(\set{n})$.
\end{propbox}

\begin{lembox}{: Hadamard identity}
    Let $x \in \set{0, 1}^k$ and the corresponding state $\ket{x} = \ket{x_1}\ket{x_2} \ldots \ket{x_k} \in \CC^{2^k}$. let $H^{\otimes k}$ be the Hadamard tensored with itself $k$ times where $H = \frac{1}{\sqrt{2}}\m{1 & 1 \\ 1 & -1}$. Then:
    \begin{equation}
        H\ket{x} = \frac{1}{\sqrt{2^k}}\sum_{y \in \set{0, 1}^k}(-1)^{x \cdot y} \ket{y}
    \end{equation}
\end{lembox}
\begin{proof}
    Left to the reader - we already did $k = 1$ when proving the phase kickback trick.
\end{proof}

\begin{lembox}{}
    Let $k, K \in \NN$. Suppose $z_1, z_2, \ldots, z_K \leftarrow \mathbb{F}^k_2$ (where $\mathbb{F}_2$ is the field of two elements, i.e $\set{0, 1}$ with XOR addition). Then, the probability that the dimension of the span of the $z_i$s, i.e. the dimension of:
    \begin{equation}
        V \coloneqq \set{a_1 z_1 + \ldots + a_Kz_K \vert \forall i a_i \in \mathbb{F}_2} \subseteq \mathbb{F}_2^k
    \end{equation} 
    is $k$ at least $1 - 2^{k-K}$. 
\end{lembox}

Sanity check: when $K \leq k$, $1 - 2^{k-K}$ is non-positive and the bound yields no information. Special case; if $K = K + 1$ then the probability is at least $1/2$